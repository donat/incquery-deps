\chapter{Introduction}
% ------------------------------------------------------------------------------
% 2 pages. give a short overview about the environment and introduce the problem
% what my solution solves.
% ------------------------------------------------------------------------------

\section{Motivation}

In today's software development practice, large projects that involve many
interrelated software components are a commonality. To tackle the complexity of
such development processes, a primary aim for both project managers and
developers is to reduce inconsistency issues that may arise due to the
combination of (i) development by large and distributed teams and (ii) the
complex dependency relationships within the system under development.

In a Java context, a typical example of such problems is when a newer
version of some component is binary incompatible with others that depend on
(link to) it. Even though linking errors can be detected at compile time in
theory, in practice this may not be always feasible as the entire source tree
may not be available to every developer.

The main challenge addressed in this work is to achieve \emph{smooth upgrades}~\cite{Ical11}.
A smooth upgrade means that after a new version of a software component has been
released, all the other software components which depend on it will remain
operational.

There are a variety of approaches and tools to support smooth upgrades. For
instance, one can enforce development process policies that involve manual
synchronization steps that all developers must follow. An alternative approach
-- perhaps better aligned with agile principles -- is to provide automated tools that
relieve the developers of the difficulties associated with software
infrastructure management.

In this work, we aim to develop such an automated tool based on static
\emph{dependency analysis}. In a few words, the goal of the tool is to assist
developers in situations such as when e.g. a bug needs to be fixed which
requires a change in a library API.
In this case, the tool should help the developer in checking which other
components are using that API, more specifically which parts of the API are
being used, which functions should remain untouched and which can be changed
freely. Using information provided by the tool, the developer should be able to
resolve such issues on her own, or in the worst case know exactly which other
team members are to be notified and called in for assistance.
 
\subsection{Controls Systems at CERN}
The context of this work is the author's internship at the Controls Group of
CERN, the European Organization for Nuclear Research. CERN runs the the world's
biggest particle physics laboratory with the main goal to operate particle
accelerators (such as the Large Hadron Collider) and all the necessary
infrastructure. At CERN, numerous scientist and engineers are working on the
design and the maintenance of the software and hardware equipment in a
well-defined structure. One of its members is the \emph{Controls Group} which is
responsible for the design and implementation of both software and hardware used
in the controls systems.

The controls system itself is a complex, distributed and highly modular system
which has three tiers. On the top there are the GUI applications which are
written in Java. The middle or business layer is also consists of Java
applications. On the lowest or hardware level there are C/C++ applications
running on real-time systems. Maintaining a complex system like this requires a
set of special approaches in order to provide desired quality and to maximize
the availability of the systems. The primary objective for the Controls Group is
to maintain uninterrupted operation without any downtime. During the internship,
the author was tasked with designing and implementing a developer tool that
should aid the software developers at CERN in minimizing software component
upgrade problems.

\subsection{Requirements specific to the deployment environment}
There are certain requirements specific to the software infrastructure at CERN
that needed to be taken into consideration while designing the system.

\begin{description}
\item[Technological environment] The software components to be managed by the
new tool are all plain Java programs that do not make use of any framework (such
as e.g. OSGi) for dependency and lifecycle management, but rely on basic
features of the Java Compiler and the Java Virtual Machine to link to each
other.
 
\item[Complexity of a large software infrastructure] In a simple case,
dependency analysis could be supported by a development environment such as
Eclipse based on integrated source/binary code analysis. However, CERN Controls
Systems deals with an infrastructure that consists of several thousands
components (JARs) and even more if versioning information is taken into
consideration. The resources available for development environments at developer
PCs are not enough for supporting standard IDE features over a system of this
size.
 
\item[Granularity of dependencies] Existing dedicated dependency analysis
tools offer a rather limited feature set for defining dependency relationships.
For example JDepend \cite{JDepend} and JBoss Tattletale \cite{Tattletale} does
dependency analysis on the class level, but CERN's requirement is to see
method-level and fields level dependencies as well.

\item[Source code analysis alone not feasible] Due to legacy reasons at CERN,
the (up-to-date) source code is not available for some software components. In
other cases, the source code is very large in size because it is automatically
generated. Therefore, tools that rely solely on source code analysis cannot be
used.

\end{description}

\section{Overview of the proposed solution}
After analysing available off-the-shelf solutions, it became clear that the
unique requirements called for a custom solution. The result of the work, called
\ptool{}, implements hybrid dependency analysis based on both source and binary
software components, and relies on a client-server architecture where end-user
features are specifically aligned with the computing resources of the host
environment. The tool uses a state-of-the-art incremental model query evaluation
engine called EMF-IncQuery (developed at the Department of Measurement and
Information Systems of the Budapest University of Technology and Economics), to
provide on-the-fly dependency analysis results integrated into the Eclipse
development tool.

\subsection{Server features}
On the server side, the \ptool{} provides the following features:
\begin{itemize}
  \item \emph{Binary dependency discovery based on byte code analysis} ensures
  that all (including legacy) software components can be taken into
  consideration for smooth upgrades, at the desired level of granularity.
  \item A \emph{scalable relational dependency database} backend ensures that
  the entire, large software infrastructure can be supported.
  \item A \emph{data access layer based on Java RMI} allows accessing dependency
  data by Java-based clients in an efficient way.
\end{itemize}

\subsection{Client features}
On the client side, the \ptool{} provides the following features:
\begin{itemize}
  \item Efficient \emph{source dependency discovery based on incremental AST
  processing}, supported by the Eclipse Java Development Tools.
  \item A \emph{model-based dependency representation for both local (source)
  and remote (binary) dependencies} based on the Eclipse Modeling Framework. The
  industry standard modeling format allows the processing of dependency data by
  third party modeling tools.
  \item \emph{On-the-fly dependency queries based on the joined local and remote
  dependency model} supported by EMF-IncQuery. As the dependency queries are
  evaluated very efficiently, the system is able to provide instantaneous
  dependency analysis feedback (relevant to both local and server-side
  information) as the source code is being changed by the developer.
  \item \emph{User interface integration components for both client-side and
  server-side queries} that allow ease of use for developers accustomed to the
  Eclipse environment.
\end{itemize}

\section{Structure of the report}
The rest of this report is structured as follows. Chapter 2 gives a brief
overview of technologies that have been used in the design and implementation.
Chapter 3 discusses the role of the tool in development workflow (Section 3.1),
provides a high-level overview of the software system architecture (Section 3.2)
and introduces a running example that is used in explanations later on (Section
3.3). Chapter 4 elaborates the design and implementation details, by discussing
components both on the server side (Section 4.1) and client side (Section 4.2).
Chapter 5 presents experimental results on the performance evaluation of the
system, and Chapter 6 concludes the report with discussing on future work.

%\todo[inline]{change for proper refs}


