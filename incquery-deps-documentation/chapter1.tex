\chapter{Introduction}
% ------------------------------------------------------------------------------
% 2 pages. give a short overview about the environment and introduce the problem
% what my solution solves.
% ------------------------------------------------------------------------------

\section{Motivation}

\todo[inline]{context}: complex software development


\subsection{Challenge: Smooth upgrades}
% What is a smooth upgrade.
The main challenge addressed in this work is to achieve \emph{smooth upgrades}. Smooth upgrade means
when a software is being updated than all the other softwares which depend on it
will remain operational. A straightforward example of braking this principle,
when the newer version is binary incompatible with dependant softwares and cause
load-time error.

\todo[inline]{preciz smooth upgrade definicio}

% How to achieve smooth upgrades.
There are a variety of tools and methods to achieve smooth upgrades. For
instance one can introduce manual procedures and protocols to the development
lifecycle. Another possibility to enforce some code metrics to achieve this
goal.

\todo[inline]{explain the connection between smooth upgrade and dependency analysis}

% One possiblity: incoming dependencies.
The last approach -- which the solution presented in this paper implements -- is
the analysis of \emph{incoming dependencies}. As an example let's say a
developer who is working a Java library which happens to be used by some
clients. Now let's say there is an error in the bug which requires a change in
the library's API for the resolution. Our developer wants to check who is using
the API, which parts are used and should remain unattached and which are the
unused parts. The users (as in code parts) are considered as incoming
dependencies. The following chapters describe an implementation for querying these 
incoming dependencies. 
 

\subsection{Control systems at CERN}

\todo[inline]{shorten}

% What is CERN.
The European Organization for Nuclear Research known as CERN is an international
organization which runs the the world's biggest particle physics laboratory.
Established at 1954 and situated on the Swiss-French border next to Geneva. The
organization's main goal to operate particle accelerators (such as the Large
Hadron Collider) and all the necessary infrastructure.

% The Controls group.
Obviously not only physicists work az CERN: numerous scientist and engineers are
working on the design and the maintenance of the software and hardware
equipments in a well-defined structure. One of its members is the Controls Group
which is responsible for the design and implementation of both software and
hardware used in the controls systems. This involves providing software
frameworks, alarm systems, communication middleware and numerous related tasks.

% Main goal for Controls group.
After the LHC became operational the  primary objective for the Controls Group
is to maintain uninterrupted operation without any downtime. This implies that a
upgrading certain parts of the systems should not interfere with arbitrary
components.

% Controls systems.
The controls systems itself is a complex, distributed and highly modular system
which has three tiers. On the top there are the GUI applications which are
written in Java. The middle or business layer is also consists of Java
applications. On the lowest or hardware level there are C/C++ applications
running on real-time systems. Maintaining a complex system like this requires a
set of special approaches in order to provide desired quality and to maximize
the availability of the systems.


\subsection{Requirements}

\todo[inline]{CERN-specific requirements}


The first valid question is: why didn't we just reuse an existing
implementation? The answer is: because there is no one.
There are specialities about the situation my tool has to solve.
\begin{description}
\item[Size limitation] If we are able to load every single jars into an IDE the
problem would be solved: we could use the internal navigation features to navigate 
through the dependency graph. In normal case seems possibe. 
But if we look at he CERN Control Systems, where we have thousands of jars (and tens of 
thousands if we count the possible versions) this approach is -- evidently -- impossible.   
\item[Fine-graininess of dependencies] The existing tools usually cannot determine
the desired dependency resolution level. For example JDepend \cite{JDepend} and
JBoss Tattletale \cite{Tattletale} does dependency analysis on the class level,
but we want to see method-level and fields level dependencies as well.
\item[No source code analysis] We don't want to do a source code analysis because 
it would imply a dependency resolution for concerned projects every time a new version 
of a product comes out. This requirement also eliminates a number of possible tools. 
\end{description}

% After considering these arguments we at CERN decided to implement our own
% solution internally called \ptool{}. In the following chapters I will present
% the design and the implementation of this tool, how is it currently applied az
% CERN Controls Group and how it was extended at Budapest University of
% Technology.


\section{Overview}

client - server approach

\subsection{server features}

\todo[inline]{itemize}

\subsection{client features}

\todo[inline]{itemize}

\section{Structure of the paper}

\todo[inline]{write me}


