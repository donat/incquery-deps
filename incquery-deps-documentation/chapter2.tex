\chapter{Background}
%------------------------------------------------------------------------------
%10 pages. For motivation: Design pattern figure. (Service registration pattern).
%------------------------------------------------------------------------------


\section{Related technologies}
% Why do we need this part.
Before proceeding to the main part of this paper, I am going to give an overview
about the related technologies used by my solution. I Why do we have to present
the related technologies.

\subsection{Java Runtime}

\subsubsection{Java byte code specification}
How the java vm works.
What is the input.
What kind of information can be extracted from here.

\subsubsection{Apache Commons Byte Code Engineering Library}
% described in the virtual machine specification 
% http://docs.oracle.com/javase/specs/jvms/se5.0/html/VMSpecTOC.doc.html 
% how the virtual machine works
% what kind of infomration is available in the bytecode 
\cite{BCEL}
Abstraction over Java byte code specification. 
How it works. Create diagram!
Possible use-cases.


\subsection{Eclipse}

\subsubsection{Eclipse Integrated Development Environment}
The Eclipse Project \cite{Eclipseproject} is an open source software development
project dedicated to providing a robust, full-featured, commercial-quality,
industry platform for the development of highly integrated tools. It was
developed by IBM from 1999, and a few months after the first version was
shipped, IBM donated the source code to the Eclipse Foundation.

The Eclipse project consists of many subprojects, the most important being the
Eclipse Platform, that defines the set of frameworks and common services that
collectively make up ,,integration-ware'' required to support a comprehensive
tool integration platform. These services and frameworks represent the common
facilities required by most tool builders, including a standard workbench user
interface and project model for managing resources, portable native widget and
user interface libraries, automatic resource delta management for incremental
compilers and builders, language-independent debug infrastructure, and
infrastructure for distributed multi-user versioned resource management.

The Eclipse Platform has an easily-extendable modular architecture, where all
functionality is achieved by plugins, running over a low-level core called
Platform Runtime. This runtime core is only responsible for loading and
connecting the available plugins, every other functionality, such as the
editors, views, project management, is handled by plugins. The plugins bundled
with Eclipse Platform include general user interface components, a common help
system for all Eclipse components, project management and team work support.

\subsubsection{Eclipse Java Development Tools}
% TODO: Write jdt description with the same details as the Eclipse platform.

\subsubsection{Eclipse Modeling Framework}

\subsubsection{EMF-IncQuery} 
Incremental model queries over emf models. Documentation site.
% https://viatra.inf.mit.bme.hu/incquery/documentation

\subsection{Other related technologies}

\subsubsection{Spring Framework}
To develop modular, testable and configurable applications.
\subsubsection{Maven}
Build system
\subsubsection{Tycho}
Maven extension to build eclipse plugins.
\subsubsection{Oracle database}
Commercial relational database management system. Full sql support. Extensive
feature set, one of the market-leaders.
% maybe it is easy to put here something from wikipedia.


\section{Example: Service Provider Framework}
\label{sect:example}

The service provider framework is a possible use-case for the Adapter design
pattern. The implementation I am going to present derived from the book
\cite{Bloch08} Effective Java.

Overview of the pattern. 
\picr{exampleclasses.pdf}{Service Provider framework example}
Three different people 

Description of the classes one-by one. Possible source code can be inserted
here.
