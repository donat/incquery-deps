\chapter{Evaluation}
%------------------------------------------------------------------------------
%3-4 pages. Functional evaluation. Performance evaluation. 
%------------------------------------------------------------------------------

\section{Example revisited}

After showing the details of the implementation, let's see it in action.
We will use the tool to solve the problem described in \autoref{sect:spf}. We
will go though on every component and check what is happening there with the
example. 

Let's start with the repositories. In our example the source and the binary
repository contains three projects, each of them holds one package from the
example. the name of the projects are service, client and impl, the name of 
the jars files respectively service.jar, client.jar and impl.jar. For the sake
of simplicity we have only one version from each project. 

\subsection{On the server side}


Now let's move forward to the server process and the dependency analysis. As 
the binary projects appear in the repository, they get discovered by the 
server process. 

The first step in the process is the bytecode analysis. In this phase the
structure and the reference of the dependencies are discovered from the
binaries. As of the structure the analyzer finds 3 projects, 8 classes 2 fields
and 19 methods which get stored in the database.
The methods are composed of the 12 defined methods, 6 constructor (from all
non-interface classes) and 1 static initializer (which comes from the Services
class as it gives a default value to a static field). 

The discovered dependencies are formatted and filtered in order not to have
internal or platform defined elements (dependencies pointing inside the
\code{java.*} package). For example let's see the \code{BasicImplUtil} class'
\code{registerImplementation()} method.
\begin{tabl}{External references of the BasicImplUtil.registerImplementation() method}{extrefs}{|p{.37\linewidth}|p{.37\linewidth}|p{.2\linewidth}|}
\hline
\textbf{Original format}	&  \textbf{Source format} & \textbf{Type} \\
\hline
impl/BasicProvider  & impl.BasicProvider &  ConstClass \\
\hline
service/Services/registerProvider \mbox{(Ljava/lang/String;} \mbox{Lservice/Provider;)V} 
&  \mbox{service.Services.registerProvider} \mbox{(java.lang.String,} \mbox{service.Provider):void} & \mbox{Constant} \mbox{MethodRef} \\
\hline
\end{tabl}
The first one is a 
\code{ConstantMethodRef} type reference. During the analysis they are first converted
The first one is a class reference the second one is a method reference.
The references are formatted to source format as it is shown in the second column.
Now the first references is eliminated from the list, because it points inside the 
\code{impl} project. However the second one does not, so it remains at its place.  

The dependency processor takes the remaining dependencies from the elements and 
searches the database if there is one. In case of our method reference, the processor
finds that there is a \code{registerProvider()} method defined with the same exact 
signature signature in the database. Because the reference was stored on the called 
method list, the processor creates a new method call dependency starting from 
\code{impl.BasicImplUtil} targeting \code{Services.registerProvider()}.

At this point the storage engine stores all the structure and the dependencies 
of the three products. The schema of the Oracle database implementation and the EMF
database was described in details previously. Now let's check how this data can be used 
on the client side.

\subsection{On the client side}

Source model synchronizer.
View of the service deveoper. as he opens the project the following emf model is
built from the state of eclipse.
\pic{wsmodel.png}{Workspace model containing Service package from the example}.
\autoref{fig:wsmodel.png} shows a readable view of the extracted model. It has
both contains all the packages classes, etc. And all internal references
\emph{including} the internal dependencies.

Workspace model synchronizer.
For the service project the entire workspace is loaded. 

Direct queries
Picture how it was queried.
Picture: result of a direct query for a field and for the method.


Model queries.
Result of the emf query.
Show the view as i
FOr example incoming dependencies.


\section{Performance evaluation}
Pure measurement. What can we measure:
the size of the database.
Speed of dependency discovery.
Size size of the compacted model. 
Memory size