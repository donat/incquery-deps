\chapter{Conclusion and Future work}
%------------------------------------------------------------------------------
%1 page. Evident content.
%------------------------------------------------------------------------------

\todo[inline]{mident bovebben kifejteni}


% what was achieved
\section{Results of the report}


\todo[inline]{sajat eredmenyek}
\begin{itemize}
  \item I designed\ldots
  \item I implemented\ldots
  \item I evaluated / experimented with\ldots
\end{itemize}

The \ptool{} is now a complete solution both with the explicit queries and its
EMF-IncQuery-based extension. 

The tool can be useful for a Java library developer who works at a large,
software-oriented organization where lots of inter-depending softwares are
developed and maintained and where binary compatibility and smooth upgrades are
mission-critical requirements.

Both parts proved useful, fast and scalable enough for adaptation. EMF-IncQuery 
showed that it can be applied to a specific domain and can be a useful core
for fast model queries. 


\paragraph*{Application of the results}
Without the extension the tool is used in
production at CERN and it is fairly possible that extension will become the part
of that tool.

\section{Future work}
As of the future plans, there are plenty of directions to go.

\paragraph*{User interface possibilities on the client side}
\begin{itemize}
  \item dedicated UI for local dependency analysis, on-the-fly synchronization
  \item Workspace/project/infrastructure-level Reference search with on-the-fly evaluation
  \item on-the-fly evaluated validation rules for instantaneous ``smooth upgrade" policy violation feedback
\end{itemize}

\paragraph*{Support for other programming languages}
The second plan can be an extension for analysing not just Java, but C++ softwares too. 
\todo[inline]{make use of Eclipse's capabilities, e.g. CDT provides an AST}

\paragraph*{Query-based software metric policy enforcement}
The third plan is
to reuse the model gathered from the project and from the workspace and then to
define certain metrics over the models. With this a source code could be
checked against certain requirements about the code quality.

\paragraph*{Dynamic dependency analysis}
First it is interesting what kind of dependencies can be extracted.
It is possible to extract more detailed dependencies from the binaries by
constructing symbolic execution traces based on the bytecode. With this, a
variety of new dependencies can be discovered.
