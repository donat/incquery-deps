\chapter{Conclusion and Future work}
%------------------------------------------------------------------------------
%1 page. Evident content.
%------------------------------------------------------------------------------

% what was achieved
The \ptool{} is now a complete solution both with the explicit queries and its
EMF-IncQuery-based extension. Without the extension the tool is used in
production at CERN and it is fairly possible that extension will become the part
of that tool.

The tool can be useful for a Java library developer who works at a large,
software-oriented organization where lots of inter-depending softwares are
developed and maintained and where binary compatibility and smooth upgrades are
mission-critical requirements.

Both parts proved useful, fast and scalable enough for adaptation. EMF-IncQuery 
showed that it can be applied to a specific domain and can be a useful core
for fast model queries. 

As of the future plans, there are plenty of directions to go.
First it is interesting what kind of dependencies can be extracted.
It is possible to extract more detailed dependencies from the binaries by
constructing symbolic execution traces based on the bytecode. With this, a
variety of new dependencies can be discovered. The second plan can be an
extension for analysing not just Java, but C++ softwares too. The third plan is
to reuse the model gathered from the project and from the workspace and then to
define certain metrics over the models. With this a source code could be
checked against certain requirements about the code quality.