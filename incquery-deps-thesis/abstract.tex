\selectlanguage{magyar}
%----------------------------------------------------------------------------
% Abstract in hungarian
%----------------------------------------------------------------------------
\chapter*{Kivonat}\addcontentsline{toc}{chapter}{Abstract in hungarian}

A svájci CERN kutatólaboratóriumban a részecskegyorsítók működtetéséhez óriási
informatikai infrastruktúrát szükséges, melyet fizikusok és mérnökök százai
fejlesztenek és üzemeltetnek. Az irányítási rendszerekben található
alkalmazások száma egyedül meghaladja az ezres nagyságrendet, melyek
karbantartatása önmagában is nagy kihívás. Mivel a teljes rendszer zavartalan
működtetése kiemelt fontosságú, ezért a leállás nélküli frissítések (smooth
upgrades) megvalósítása alapvető elvárás.

A rendszereket elsősorban – a hardver közeli komponenseken kívül – tisztán Java
nyelven fejlesztik. A szoftverek API-kat definiálnak (adatgyűjtés, megbízható
üzenetküldés, könyvtárszolgáltatás, stb.), melyet más szoftverek
felhasználhatnak. Az fejlesztéskor a legfontosabb azonosított feladat, hogy a
ráépülő elemek binárisan kompatibilisek legyenek az egyes szoftverek újabb és
újabb verzióval.

Elsődleges feladatom, hogy megvizsgáljam milyen módszerek és eszközök
segítségével lehet a rendszerek közötti függőségeket analizálni. Megvizsgálom,
hogy milyen típusú függőségek jelennek meg az alkalmazások között, ezeknek
milyen tulajdonságai vannak és hogyan lehet ezeket hatékonyan lekérdezni.
Megtervezek és implementálok egy olyan szoftvert, mely a fejlesztőkörnyezetbe
integrált módon képes a függőségeken lekérdezéseket végrehajtani, valamint
képes az eredmények grafikus megjelenítésére. Emellett megvizsgálom, milyen
egyéb módszerek állnak rendelkezésre, hogy az egyes programok közötti
függőségek definiálására és felderítésére, mely később a fejlesztők munkáját
segítheti.
\vfill

\selectlanguage{english}
%----------------------------------------------------------------------------
% Abstract in english
%----------------------------------------------------------------------------
\chapter*{Abstract}\addcontentsline{toc}{chapter}{Abstract in english}
Maintaining the particle accelerators is CERN research laboratory requires a
huge informatics infrastructure developed and maintained by hundreds of
physicists and engineers. The number of the controls systems itself exceeds the
thousand, which needs a big effort to operate. Since the uninterrupted
operation of the whole system is a priority task, the developers' primary task
is to ensure the smooth upgrades.

The softwares – apart from the hardware-related components – are primarily
developed in pure Java language. The softwares define an API (for e.g. data
acquisition, reliable messaging, directory service, etc.), which could be used
by other applications. At development time the most important identified task 
to maintain binary backward compatibility between de dependent component and 
the upgraded software through the newer versions.

My primary goal is to examine, which techniques and tools are available to analyse 
the dependency relationships between the softwares. I will investigate what type 
of dependencies are present in the systems, what properties do they have and how 
can one query them easily. I will design and implement a tool, which is capable
 of querying and visualizing the dependencies integrated into the development 
 environment. In addition I will examine what other methods are available for 
 defining and discovering the dependencies, which could be useful for the 

\vfill

