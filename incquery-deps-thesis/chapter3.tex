%
% ------------------------------------------------------------------------------ 3 pages. Super figure which describes the entire solution. Choosing proper
% abstraction level for the %figure is essential. Every item on this figure will
% be an additional chapter in this paper.
% TODO:find a proper name for this chapter.
% ------------------------------------------------------------------------------
\chapter{Overview of the approach}

% My implementation. 
One of the most effective  way of doing
smooth upgrades is to discover incoming dependencies. To achieve this, we
designed and implemented the \ptool{}. The the first part of the current chapter
introduces the environment where the tool is deployed and used. The second part
gives an overview on the \ptool{} approach by describing the objectives of the
contained components. In the last part a running example is introduced to make
the solution more understandable.


\section{CERN software infrastructure and processes}\label{sect:cerninf}

\subsection{Software infrastructure}

% Eclipse
The CERN Controls group has a definite set of tools used for development. Because
C/C++ and Java applications are developed, the used Integrated Development
Environment is an internally maintained Eclipse distribution with correct
plugins installed.

% SVN, JIRA, Bamboo
To maintain concurrent versions of the source code an SVN server is used. Also
there are a variety of tools used for typical developments tasks: issue
etc. 

% Common Build
The central element for the development is a tool called \tool{Common Build}
\cite{CommonBuild}. It is a build and release tool for Java softwares. Common
Build is an Apache Ant based software similar to Maven. It provides
functionality to describe and resolve dependencies, build, generate
documentation and release the softwares. This tool was developed before the
first version of complex build systems such as Apache Maven was published so,
as a result 
Common Build remained as an in-house build tool. 

% Common build integration with source and binary repositories
Common build is tightly coupled both to the SVN server and to the binary release
repository as it automates the not only the building but the release of the
Java products too.

\subsection{Development workflow}
\picseventy{commonbuild.pdf}{Build workflow using Common Build}
\autoref{fig:commonbuild.pdf} highlights this relationship by showing a typical
workflow of the development process. The developer first checks out the source
code from the SVN repository and starts working on it. Along source code
modification there is an XML descriptor containing the information required by
Common Build (such as name and version of the product and the required
dependencies). Traditionally Common Build is capable to setup build paths and the related
options for Eclipse development. 

\subsubsection{Build process}
When the source code is ready, the developer executes the Common Build client to
build it. The client itself is a customized Ant distribution and works in the
same way. It resolves the dependencies from the dedicated binary repository,
which contains all previously released products and all used third party
libraries. The content is also defined in XML file format.

\subsubsection{Release process}
If all source code is ready the developer can hit release. This first initiates
the same build process, but it is extended by two things. First the source code
is tagged in the SVN repository using the version numbers as tag names, then, the
compiled jar file is put into the binary repository. 

\subsubsection{Smooth upgrades and the development workflow}
On \autoref{fig:commonbuild.pdf}, smooth upgrades should be supported during working
with the Eclipse IDE (the arrow with label ''2. modification''), in the pre-release phase when the
developer is making local changes with the aim to ensure that these changes will
not break existing software components already released.


% The developers at the controls group work on a variety of Java softwares, which
% have complex relationships. Mostly these softwares has to stay operational 24/7.
% In addition an upgrade or a bugfix is inevitable, sometimes instantly. But if a
% software changes and it breaks the API, a dependant deployed software may stop
% working and in worst case the controls systems stop. Because of this, the
% developers have to perform their upgrade carefully without causing damage to the
% dependant systems.
% 
% The initial solution for a developer was to search the source repository for specific
% texts, which is ineffective and can miss certain use-cases. This is why my project
% was started.
% 
% Right now the tool is distributed with an internal eclipse packaging, so it is available 
% for every developer who needs it. Usually library developers take advantage of it 
% especially during technical stops, where the scheduled upgrades take place. 


%kifejteni, hogy hogyan kapcsolodik pontosan a smooth upgrade a fejlesztesi folyamatba}
 
%ki, mikor, mire, hogyan hasznalja az altalad elkeszitett toolt?}

  

\section{System architecture}

\autoref{fig:superfigure.pdf} shows the architecture of the \ptool{}.
\picr{superfigure.pdf}{Overview of the implemented system}
The system applies the standard server-client
architecture. The server side discovers the Java binaries, processes and stores
their structure and stores them in a database. The client side is an Eclipse
plugin, which gives the ability to the users to obtain and analyse the
dependency information about the softwares loaded into the workspace. 


\subsection{System boundaries}
The system allows the users to examine the inter-project dependencies from the
development environment. It exposes two access points to access this
information. First the developer can initiate queries explicitly by selection
the target element from  he source code editor. The other option to check result
of the model queries (described later). This information is updated
automatically, every change in the development environment is reflected to the
result set. Both versions give visual representation of the result which the
developer can analyze and he can use it to make decisions what to and what not
to change in the source code.


\subsection{Repository management}
% Central repository management.
This part consists of the ''Source repository'' and the ''Binary component
repository''. They contain the source code and the built binary version of
the internally developed softwares. These elements are centrally managed and
have a well-defined structure. The developers work on the source code and if
they finish one milestone they publish their improvements by putting the
compiled version into the binary repository through an automated process.
The binary repository is the input for the dependency analysis approach.

\subsection{The server side}
The server side is a standalone Java application which runs on a Linux server.
It has two functionality: (i) it discovers and stores the structure of the
products and (ii) provides an interface for serving dependency queries. It has three
elements, the bytecode analyser, the dependency processor, and the storage engine.
The components of the server side are explained in detail in \autoref{sect:elabserver}.

\subsubsection{Bytecode analyser}
When a new binary jar file is discovered the bytecode analyzer  discovers the
internal structure and the dependency references. The result of the bytecode
analyzer is an instance model representing the Java binary structure.

\subsubsection{Dependency processing}
The dependency processor takes the output of the bytecode analyzer, stores the 
structure of the Java binaries and tries to resolve the external references.
For both storing the structure and resolving the dependencies, this component
executes several queries on the dependency database. As a result, all structural
information are stored in this database, which serves the client queries too.


\subsection{The client side}
% Plugin
The client side of the solution is a set of Eclipse plugins which give the
developer a convenient way to access to the dependency information. 

\subsubsection{Model load and explicit queries}\label{sect:directqueries}
The base component of the client side is the repository model loader. It
provides a simple API for accessing and querying the dependency information from
the server. The simple use-case for this, when the user directly asks the
dependency information from the Java source editor through UI contribution
(marked as ''explicit queries'' on the figure).

\subsubsection{Workspace model creator}\label{sect:wsmodelcreator}
The workspace model creator generates an EMF instance model describing the
state of the workspace. The generated model contains the loaded projects, the
declared packages, classes, methods, etc. This instance model also contains the
dependency structure between the elements (e.g. method calls, field accesses and
such). The information is gathered through the Eclipse Java Development Tools'
API. After the model is created it is incrementally maintained even through 
restarts.

\subsubsection{On-the-fly queries}
The pattern matcher module executes complex model queries on top of the acquired
instance models to provide dependency information. First, it loads an EMF
instance model from the server describing the structure of all projects in the
central repository. Afterward the workspace instance model is loaded from the
module (see \autoref{sect:wsmodelcreator}). Both of the models are loaded to the
EMF-IncQuery engine. Through complex queries the two models are joined and
queried for the dependency information. The result of the queries are maintained
live and automatically as the workspace instance model changes. Because the
state of the workspace is followed, more information can be queried than just
comparing the source code with the state of the repositories
(\autoref{sect:directqueries}): The source code changes and their effects
are also queried.

\subsubsection{User Interface}
Both direct- and on-the-fly queries return their result in Eclipse views. After
the result is displayed, it is the user's responsibility to evaluate their
results and act accordingly. The user interface consist of the following.
\begin{itemize}
  \item Menu contributions: The Java source code editor gets a new elements for 
initiating dependency discovery.
  \item Result viewers: The result is displayed in Eclipse views.
  \item Source code markers: Important model validations result in an error marker 
  in the source code editor.
\end{itemize}
The details of the user interface implementation are in \autoref{sect:elabexplicit}.

\subsubsection{Implementation-specific requirements}
Using explicit queries brings some limitations. The queries are simple remote method
calls and the result set is a relatively small data set which easy to store and
present on the Eclipse UI. On the other hand, the graph-pattern matching
solution is somewhat limited, because in order to make EMF-IncQuery work,
the entire repository has to be loaded. By default the repository instance model
is a few hundred megabytes sized in a serialized form. This implies that the
implementation has to optimize this model without dropping useful information.

\section{Running example: Java API for Parameter Control}\label{sect:spf}
To make the following chapters easier to understand, we are going to introduce
a simplified version of a real-life example from the CERN controls systems called
\emph{Java API for parameter Control (JAPC)}\cite{Japc05}.  JAPC is a
framework to build Java applications that controls accelerator devices by
providing a unified API for it. 

\subsection{Details}\label{sect:spfdesc}

\subsubsection{Motivation}
The main idea behind JAPC is that there are a different requirements for the 
softwares living in the application and the resource (hardware) level. 

On the hardware level there are control systems, data acquisitions 
running on specialized hardware which comes with language and resource 
constraints. On the application site, the main requirement is the
maintainability: softwares should not depend on technology
specific-implementation: it has to be easily modifiable and upgradeable.

\subsubsection{Architecture}
\picfifty{japcarchitecture.pdf}{The architecture of JAPC}

\autoref{fig:japcarchitecture.pdf} shows the architecture of JAPC. The concept
of the framework is fairly simple: provide a unified API for accessing and
modifying data on certain resources around the controls systems. 

The implementation of the client applications has to use only the API defined
in the core JAPC project. The details are contained in a separate extension
libraries. They hold the  technology-specific details for accessing and
modifying the parameters. 

To use JAPC, the developer has to include the core API and the extension
library during the build. The connection between the two is driven externally (through runtime
configuration and registration).


 \subsubsection{Class hierarchy}
JAPC's API is rather complex because it has to fulfill a wide range of
requirements. For sake of brevity we are going to show the sub-part of it
as it is enough to demonstrate how the dependency analysis work.


\picseventy{japcclasses.pdf}{Simplified version of JAPC's application
programming interface}

\autoref{fig:japcclasses.pdf} shows the class hierarchy.
The central element of the API is the \code{Parameter} interface.
It  represents an abstract value of a device.
Client programs can access JAPC parameters with set and get interactions, and
they can wait to be notified of value changes using subscription.

The \code{Parameter} gets and sets its value through a \code{ParameterValue} instance.
Instantiation of a \code{Parameter} instance is done via the
\code{ParameterFactory} interface's \code{newParameter()} method. Again the
implementation of these elements are defined externally.

\subsubsection{Examined projects in the example}
In the elaboration of the example (\autoref{sect:elabex1} and
\autoref{sect:elabex2}) we are going to use three simplified projects to
demonstrate how the dependency discovery works and how can the users benefit
from using the presented tool.

The first project is the JAPC core API (shown on \autoref{fig:japcclasses.pdf}) called \code{japc}.
The second one is the \code{japc-ext-cmw} a JAPC extension library representing an implementation based on 
an in-house middleware technology (Controls Middleware Project\cite{Cmw}). The
third one is a client application called \code{accsoft-gui-alarms}, which has a JAPC-based implementation and has several deployments around the CERN Controls Systems.


\subsection{Smooth upgrade challenges}
As our main goal we would like give the developers the ability to check who is
using certain parts of the API. To highlight this, imagine that the three
projects described in \autoref{sect:spfdesc} are maintained by three individual developers. 
If the developer
responsible for the \code{japc} project wants to change some parts of the
code without precisely knowing who is using which part of the code, he won't
know how many dependent projects will be broken afterwards. 

By using \ptool{} the developer is able to execute queries on any part of the
API. as a result three outcome can happen.
\begin{description}
  \item[No incoming dependencies] If there is no incoming dependency than any modifications can be done without
causing compatibility issues. The developer can do whatever he wants.
  \item[A few incoming dependencies] In this case the developer has to negotiate
with the clients how to proceed: either he has to provide backward compatibility or the
clients has to adapt their code to the modifications.
  \item[A lot of incoming dependencies] In this case the queried code
element is used by a lot of external clients and any modifications would cause a build
error, therefore backward compatibility has to be maintained.
\end{description}
The elaboration of this example is in \autoref{sect:elabex1} and in \autoref{sect:elabex2}.
