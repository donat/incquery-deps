\chapter{Performance evaluation}
%------------------------------------------------------------------------------
%3-4 pages. Performance evaluation. 
%------------------------------------------------------------------------------

After the functionality was presented, let's check the overall performance of
the system. We have to examine the individual components that whether fulfill their
requirement as the input or usage scales up to a level of the production
applications. First we have to verify whether the dependency analysis is fast
enough and can process large number of Java binaries. The next task is to see if
the explicit queries are reasonably fast. The last yet the most important is to
test to performance of the EMF-IncQuery engine by checking how big instance
models can be loaded.

The JAR files used as the sample data come from real-life applications; they are
operational software components from the CERN controls systems. By using them we
can eliminate the problems of having badly generated, homogeneous and
unrealistic sample data.

\section{Dependency processing}
\label{sect:depproc}
The first task to measure is how the server process performs.
A measurement has to check how efficiency the bytecode analysis and the
dependency processing together works. A minimal preliminary requirement is that
it should be much faster than the pace at which the source code in the
repository changes, because otherwise the dependency database will contain
constantly outdated information.

As a test we took all of the import binaries, started the server process as a
standalone Java application, initiated the dependency discovery on the subset of
the binaries and measured the statistics of the database and the speed of the
discovery. After the test has finished, we erased the database and repeated the
test with a bigger and bigger subset of the JAR files until the entire set was
analyzed. For the test we used a desktop computer with the following
specifications:
\begin{itemize}
  \item CPU: Intel Core i3-2120, 3.3 GHz,
  \item RAM: 8GB DDR3,
  \item OS: Windows 7 Enterprise, 64 bit, Service Pack 1,
  \item JVM: Oracle JDK 1.6.0\_27-b07.
\end{itemize}
The tests showed that on average one project contains 80 classes, 513 methods, 304 fields and 778 dependencies. 
The results of the tests are shown on \autoref{fig:resanal.pdf}
\picr{resanal.pdf}{Result of dependency processing measurement}
The left chart shows how big the generated EMF instance model becomes, depending
on the input JAR size. It is clearly visible that the size linearly increases as
the input grows. Also the execution time (right chart) shows steady growth as
the input size increases. Looking at the analysis
 of all (1300) jars, the average processing takes roughly $0.43$ seconds.
Knowing that even in large software repositories usually a few, maybe a few
dozen releases happen, then having a system which analyzes the differences
within a second is more than enough.

\section{Explicit queries}
\label{sect:explqueries}
Next question is how fast are the explicit queries. We want to see that the
results of a dependency query returns the data and visualizes the information
within reasonable time even when the result set is bigger than usual. 

We set up the server process on the same computer that we used for testing the
dependency processing. We filled the database with with all the dependency
information. For measuring the speed of the queries, we used an typical
developer virtual machine from the CERN ecosystem:
\begin{itemize}
  \item CPU: Intel Xeon E5645, 2.4 GHz,
  \item RAM: 4GB,
  \item OS: Windows 7 Enterprise, 32 bit, Service Pack 1,
  \item JVM: Oracle JDK 1.6.0\_35-b10.
\end{itemize}
Because this machine runs on a virtualized infrastructure it can't be considered
as a steady platform; performance could change as the usage changes, but it is
resourceful enough for comfortable Java development.

For the tests we chose a specific project which happens to be a widely-used
library by the other JARs. During the measurement a query is initiated for 
the incoming dependencies for every single element of this project. 
We dropped the result where there were no dependencies and accumulated the
results to show how the size of the result set affects on the response time.
The results are shown on \autoref{fig:expqmeasure.pdf}.
 \picfifty{expqmeasure.pdf}{Explicit queries measurement result}
As it turned out, the response time does not depend on the size of the
result, as the result was returned in about 200ms on every sample.

It is important to note that the measurement does not include the network delay.
Also the returned returned results contain dependency information \emph{for one
single element}. Overall, however, this is considered an acceptable response
time for retrieving (individual) dependency information.


\section{Model queries}
Using EMF-IncQuery the question is always about the memory usage. If the model
and the query caches fit into RAM, then the update mechanism requires minimal
resources to keep query results consistent with the (changing) model.

So, to test model queries we reused the models from the previous tests (see
\autoref{sect:depproc}), and generated the corresponding compacted models from
them.
The description of the compacted model is at \autoref{sect:depdbsynch}. Normally
this model would load automatically from the server side, but now we want to see
how large models could be loaded at once to judge the practical feasibility of
our approach.

The test environment is the same virtual PC introduced in
\autoref{sect:explqueries}. Because this is a typical developer machine, it is
also considered as a reference for the usability. In the measurement we opened
an Eclipse instance, loaded the patterns and made some modifications in the
workspace. We checked how fast the EMF-IncQuery initializes (including loading
the model), how much memory the entire Eclipse instance consumes and afterwards
how fast can it react to the workspace modifications.
\picr{modelqueries.pdf}{Model queries measurement result}

The result of the measurement is shown on \autoref{fig:modelqueries.pdf}.
The query engine initialized relatively fast, even the model which holds 400
projects could load within a minute. Considering that it has to be done once
per working session it is acceptable. Also with the memory allocation was linear
with the model size, it didn't explode, grew linearly with the input size.
No models bigger than 400 elements could fit into the memory because of the 32
bit platform limitation (64 bit systems could, naturally, extend this range).
The response times for query updates after model modifications were always
instantaneous (regardless of query or model size), which is consistent with
literature results reported in~\cite{MODELS10}.

\section{Performance evaluation conclusions}
In summary, the performance evaluation of the system has concluded with the following.
\begin{description}
\item[On the server side] the system provides fast analysis, persistence and
queries for the global (binary) network of dependencies with full granularity.
\item[On the client side] the system is able to support on-the-fly dependency
analysis, with full granularity on the local (source) scope and limited
granularity on the union of the local and global scopes (which is an acceptable
tradeoff for performance and resource utilization).
\end{description}

It is also important to note that the 400 project workload is an
over-approximation of what is actually needed in practice. The reason for this
is that in each session, a developer actually only needs a small subset of the
400 projects to work with as the high-level dependency network between CERN
projects is relatively sparse (this assumption presumably holds for other
typical organizations as well). Overall, all requirements could be fulfilled
deeming the approach suitable for deployment.

% In conclusion we saw that all components have the desired effectiveness factor.
% The system itself is usable, scalable with the input and a feasible solution for
% achieving smooth upgrades. 
