\chapter{Conclusions and future work}
%------------------------------------------------------------------------------
%1 page. Evident content.
%------------------------------------------------------------------------------


\section{Results of the report}

In this paper we proposed an effective dependency analysis approach for
supporting smooth upgrades in large Java software infrastructure consisting of
tens of thousands of classes. It is based on a two-tiered approach, where the
server side is responsible for accumulating, storing and processing the incoming
dependency relations between the different API elements of the complete software
infrastructure, while the client side supports the developer by providing
instantaneous feedback on the dependency relations between the software modules
currently under development and the overall software infrastructure.

As a summary, my results in this work are the following:
\begin{itemize}
\item I designed a \emph{client-server based architecture for incoming
dependency analysis} of large Java software infrastructures.
\item I developed a \emph{binary dependency discovery module} for finding
dependencies between API elements within JARs.
\item I implemented a \emph{storage system} for the dependency relations and an
access layer for querying the dependency information.
\item I proposed a \emph{model-based dependency representation} based on EMF for
capturing local and remote dependencies.
\item I defined an \emph{incremental AST processing module} for discovering
source dependencies in the local workspace.
\item I implemented an \emph{on-the-fly dependency query engine} using
EMF-IncQuery for instantaneous dependency analysis feedback.
\item Finally, I evaluated the approach on the \emph{complete Java software
infrastructure at CERN} consisting of more than 1300 separate JARs.
\end{itemize}

%The \ptool{} is now a complete solution both with the explicit queries and its
%EMF-IncQuery-based extension. 

%The tool can be useful for a Java library developer who works at a large,
%software-oriented organization where lots of inter-depending softwares are
% developed and maintained and where binary compatibility and smooth upgrades are
% mission-critical requirements.
% 
% Both parts proved useful, fast and scalable enough for adaptation. EMF-IncQuery 
% showed that it can be applied to a specific domain and can be a useful core
% for fast model queries. 

\paragraph*{Application of the results}
As a major achievement, the server side modules of the proposed approach are
already used in production at CERN and its client side extension is also being
considered to be included in the upcoming release of the \ptool.

\section{Future work}
As of the future plans, there are plenty of directions to extend the
capabilities of the approach:
\begin{itemize}
\item In order to provide a better user notification on the client side we plan
to create a dedicated UI for highlighting the result of the local dependency
analysis. As Eclipse already provides a facility for searching for cross
compilation unit references, it is a straightforward task to integrate the model
queries there. Additionally, due to the instantaneous evaluation performance,
dependency analysis results could be displayed to the user as warnings or even
errors (supported by inline markers in Eclipse's source code editors) that
appear on-the-fly as the user is making a change that potentially violates
smooth upgrade policies.

%\item on-the-fly evaluated validation rules for instantaneous ``smooth upgrade" policy violation feedback

\item The second plan can be an extension for source-analyzing not just Java,
but C++ programs too. We could make use of the capabilities of the Eclipse CDT
to provide AST support.

\item As a future improvement, we plan to support the definition of query-based
software metrics and use our on-the-fly query evaluation engine to enforce these
policies directly on the AST models.

\item Finally, an additional step forward would be the use of dynamic dependency
analysis techniques such as symbolic execution traces that would provide more
detailed dependency information between the methods of the various modules.
\end{itemize}
